%%%%%%%%%%%%%%%%%%%%%%%%%%%%%%%%%%%%%%%%%%%%%%%%%%%%%%%%%%%%%%%%%%%%%%%
% Universidade Federal de Santa Catarina
% Biblioteca Universitária
%----------------------------------------------------------------------
% Exemplo de utilização da documentclass ufscThesis
%----------------------------------------------------------------------
% (c)2013 Roberto Simoni (roberto.emc@gmail.com)
%         Carlos R Rocha (cticarlo@gmail.com)
%         Rafael M Casali (rafaelmcasali@yahoo.com.br)
%%%%%%%%%%%%%%%%%%%%%%%%%%%%%%%%%%%%%%%%%%%%%%%%%%%%%%%%%%%%%%%%%%%%%%%
\documentclass{ufsc-thesis} % Definicao do documentclass ufscThesis

\usepackage{abntex2cite}

%----------------------------------------------------------------------
% Pacotes usados especificamente neste documento
\usepackage{graphicx}
\usepackage{color}
\usepackage{listings}
\usepackage{multirow}
\usepackage{tabularx}
\usepackage[table]{xcolor}
\usepackage{colortbl}
\usepackage{framed}
\usepackage{footnote}
\makesavenoteenv{tabular}
\makesavenoteenv{table}
\usepackage[printonlyused, withpage]{acronym}
\usepackage{pdfpages}
\usepackage{afterpage}
\usepackage{hhline}
\usepackage{enumitem}
\usepackage{latexsym}
\usepackage{bm}
\usepackage{amssymb}
\usepackage[normalem]{ulem}
 \usepackage[lined,boxed,ruled,commentsnumbered,portuguese]{algorithm2e}

\usepackage{tikz}
\usetikzlibrary{decorations.pathreplacing}
\usepackage{standalone}

%----------------------------------------------------------------------
% Comandos criados pelo usuário
\newcommand{\afazer}[1]{{\color{red}{#1}}}
\newcommand{\critico}[1]{{\color{red}\textbf{{#1}}}}
\newcommand{\crit}[1]{{\color{red}\textbf{\uppercase{{#1}}}}}

\definecolor{shadecolor}{rgb}{0.8,0.8,0.8}
\newcommand\VRule[1][\arrayrulewidth]{\vrule width #1}

%----------------------------------------------------------------------
% Identificadores do trabalho
% Usados para preencher os elementos pré-textuais
\titulo{Verificação presilício de memória baseada em testes dirigidos adaptativos}
\autor{Marleson Graf}
\data{\today}
\instituicao{Universidade Federal de Santa Catarina}
\local{Florianópolis} % Opcional (Florianópolis é o padrão)
\tipotrabalho{Trabalho de Conclus\~{a}o de Curso}
\orientador{Prof. Dr. Luiz Claudio Villar dos Santos}
\programa{Curso de Bacharelado em Ciências da Com\-pu\-ta\-ção}

\centro{Departamento de Informática e Estatística}
% \grau{Bacharel em Ciências da Computação}

% Preâmbulo
% \titulo{Exemplo do uso da class \abnTeX}
% \autor{Mateus {}Dubiela Oliveira}
% \data{\today}
% \instituicao{Universidade Federal de Santa Catarina}
% \local{Florianópolis, SC-Brasil}
% \tipotrabalho{Exemplo para referência futura}
% \orientador{Pedro Henrique {}de Souza}

\coorientador{José {}da Silva Sauro}
% \programa{Programa de Pós-Graduação em Ciências da Computação}
\preambulo{Modelo can\^{o}nico de trabalho de conclusão de curso para acadêmicos da UFSC e usuários da plataforma \abnTeX.}
% \centro{Centro Tecnológico -- CTC}
\assuntos{Ciências da Computação,Modelos,Teses,OpenSource,LaTeX}


%\numerodemembrosnabanca{3}
% \orientadornabanca{sim}
%\coorientadornabanca{sim}
%\bancaMembroA{Primeiro membro\\Universidade ...}
% \bancaMembroB{Prof. Dr. Djones Vinicius Lettnin\\Universidade Federal de Santa Catarina}
% \bancaMembroC{Prof. Dr. Laércio Lima Pilla\\Universidade Federal de Santa Catarina}

%\dedicatoria{Este trabalho é dedicado aos meus colegas de classe e aos meus queridos pais.}

%\agradecimento{Inserir os agradecimentos aos colaboradores à execução do trabalho.}

%\epigrafe{Texto da Epígrafe. Citação relativa ao tema do trabalho. É opcional. A epígrafe pode também aparecer na abertura de cada seção ou capítulo.}
%{(Autor da epígrafe, ano)}

% \textoResumo {%

% }

\assuntos{EDA, verificação, pré-silício, memória compartilhada, multicore, coerência de cache}

%\textoResumo {O presente TCC visa desenvolver o projeto de blocos aceleradores em hardware, de alta eficiência energética, para a etapa de \ac{me} na codificação de vídeo de alta resolução e compatível com o padrão de \ac{hevc}. Serão estudadas soluções que representam o estado da arte para \ac{me} com tamanho de bloco variável, com ênfase na métrica de similaridade conhecida por \ac{satd}, a qual sabe-se resultar em melhor qualidade de codificação. Ainda, serão identificadas oportunidades de otimizações de eficiência energética, devendo estar associadas à solução arquitetural e ao uso de técnicas de \textit{low-power}.}
%\palavrasChave {Palavra-chave 1. Palavra-chave 2.  Palavra-chave 3. }

%\textAbstract {Resumo traduzido para outros idiomas, neste caso, inglês. Segue o formato do resumo feito na língua vernácula. As palavras-chave traduzidas, versão em língua estrangeira, são colocadas abaixo do texto precedidas pela expressão ``Keywords'', separadas por ponto.}
%\keywords {Keyword 1. Keyword 2. Keyword 3.}

%--------------------------------------------------------------------------
% Início do documento

\begin{document}
%--------------------------------------------------------------------------
% Elementos pré-textuais
\pretextual%
\imprimircapa

%\folhaderosto[comficha] % Se nao quiser imprimir a ficha, é só não usar o parâmetro
\imprimirfolhaderosto

\afterpage{\null\newpage}

\clearpage
\newpage

%--------------------------------------------------------------------------
% folha de aprovação de proposta de TCC

\begin{snugshade}

\begin{center}
{\textbf{\small{FOLHA DE APROVAÇÃO DE PROPOSTA DE TCC}}}

\end{center}

\end{snugshade}

% \vspace{-8pt}
% \small
\noindent\resizebox{\textwidth}{!}{
	\begin{tabular}{|l|X p{8cm}|}
		%\begin{small}
			\hline
   \textbf{Acadêmico} &  \imprimirautor \\ \hline
   \textbf{Título do trabalho} & \imprimirtitulo \\ \hline
   \textbf{Curso} & Ciências da Computação/INE/UFSC \\ \hline
   \textbf{Área de Concentração} &  \textit{Hardware} \\ \hline

		%\end{small}
	\end{tabular}
}

\vspace{8pt}

\noindent\textbf{Instruções para preenchimento pelo \uline{\small{ORIENTADOR DO TRABALHO}}:}
\begin{itemize}[leftmargin=*,noitemsep,topsep=0pt]
	\item[-] \small Para cada critério avaliado, assinale um X na coluna SIM apenas se considerado aprovado. Caso contrário, indique as alterações necessárias na coluna Observação.
\end{itemize}

\vspace{8pt}

\noindent\resizebox{\textwidth}{!}{
	\begin{tabular}{|X p{6cm}|X p{0.5cm}|X p{0.8cm}|X p{0.5cm}|X p{1cm}|X p{3.2cm}|}
			\hline
   \cellcolor{shadecolor} & \multicolumn{4}{c|}{\cellcolor{shadecolor} \textbf{Aprovado}} & \cellcolor{shadecolor} \\ \hhline{*{1}{>{\arrayrulecolor{shadecolor}}-}*{4}{>{\arrayrulecolor{black}}|-}>{\arrayrulecolor{shadecolor}}|->{\arrayrulecolor{black}}}
   \multirow{-1}{*}{\cellcolor{shadecolor} \textbf{Critérios}} & \cellcolor{shadecolor}\textbf{Sim} &  \cellcolor{shadecolor}\textbf{Parcial} &  \cellcolor{shadecolor}\textbf{Não} &  \cellcolor{shadecolor}\textbf{Não se aplica} & \multirow{-1}{*}{\cellcolor{shadecolor} \textbf{Observação}} \\ \hline
			{\small 1.O trabalho é adequado para um TCC no CCO/SIN (relevância/abrangência)?} & \cellcolor{shadecolor}  & \cellcolor{shadecolor}  & \cellcolor{shadecolor}  & \cellcolor{shadecolor}  & \\ \hline
			{\small 2.O título do trabalho é adequado?} & \cellcolor{shadecolor}  & \cellcolor{shadecolor}  & \cellcolor{shadecolor}  & \cellcolor{shadecolor}  & \\ \hline
			{\small 3.O tema de pesquisa está claramente descrito?} & \cellcolor{shadecolor}  & \cellcolor{shadecolor}  & \cellcolor{shadecolor}  & \cellcolor{shadecolor}  & \\ \hline
			{\small 4.O problema/hipóteses de pesquisa do trabalho está claramente identificado?} & \cellcolor{shadecolor}  & \cellcolor{shadecolor}  & \cellcolor{shadecolor}  & \cellcolor{shadecolor}  & \\ \hline
			{\small 5.A relevância da pesquisa é justificada?} & \cellcolor{shadecolor}  & \cellcolor{shadecolor}  & \cellcolor{shadecolor}  & \cellcolor{shadecolor}  & \\ \hline
			{\small 6.Os objetivos descrevem completa e claramente o que se pretende alcançar neste trabalho?} & \cellcolor{shadecolor}  & \cellcolor{shadecolor}  & \cellcolor{shadecolor}  & \cellcolor{shadecolor}  & \\ \hline
			{\small 7.É definido o método a ser adotado no trabalho? O método condiz com os objetivos e é adequado para um TCC? } & \cellcolor{shadecolor}  & \cellcolor{shadecolor}  & \cellcolor{shadecolor}  & \cellcolor{shadecolor}  & \\ \hline
			{\small 8.Foi definido um cronograma coerente com o método definido (indicando todas as atividades) e com as datas das entregas (p.ex.Projeto I, II, Defesa)?} & \cellcolor{shadecolor}  & \cellcolor{shadecolor}  & \cellcolor{shadecolor}  & \cellcolor{shadecolor}  & \\ \hline
			{\small 9.Foram identificados custos relativos à execução deste trabalho (se houver)? Haverá financiamento para estes custos?} & \cellcolor{shadecolor}  & \cellcolor{shadecolor}  & \cellcolor{shadecolor}  & \cellcolor{shadecolor}  & \\ \hline
			{\small 10.Foram identificados todos os envolvidos neste trabalho?} & \cellcolor{shadecolor}  & \cellcolor{shadecolor}  & \cellcolor{shadecolor}  & \cellcolor{shadecolor}  & \\ \hline
			{\small 11.As formas de comunicação foram definidas (ex: horários para orientação)?} & \cellcolor{shadecolor}  & \cellcolor{shadecolor}  & \cellcolor{shadecolor}  & \cellcolor{shadecolor}  & \\ \hline
			{\small 12.Riscos potenciais que podem causar desvios do plano foram identificados?} & \cellcolor{shadecolor}  & \cellcolor{shadecolor}  & \cellcolor{shadecolor}  & \cellcolor{shadecolor}  & \\ \hline
			{\small 13.Caso o TCC envolva a produção de um software ou outro tipo de produto e seja desenvolvido também como uma atividade	realizada numa empresa ou laboratório, consta da proposta uma declaração (Anexo 3) de ciência e concordância com a entrega do código fonte e/ou documentação produzidos? } & \cellcolor{shadecolor}  & \cellcolor{shadecolor}  & \cellcolor{shadecolor}  & \cellcolor{shadecolor}  & \\ \hline

		%\end{small}
	\end{tabular}
}

\vspace{12pt}


\noindent\resizebox{\textwidth}{!}{
	\begin{tabular}{|X p{3cm}|X p{2.35cm}|X p{1.6cm}|X p{3.4cm}|}
			\hline
   {\footnotesize \textbf{Avaliação}} &  \multicolumn{1}{l}{\textbf{$\Box$ \footnotesize Aprovado}}  & \multicolumn{2}{c|}{\textbf{$\Box$ \footnotesize Não Aprovado}}  \\ \hline \hline

   {\footnotesize \textbf{Professor Responsável}} &  {\footnotesize Prof. Dr. Luiz Claudio Villar dos Santos} &   & \\ \hline
   {\footnotesize \textbf{Orientador}} &  {\footnotesize Prof. Dr. Luiz Claudio Villar dos Santos} &  & \\ \hline
	\end{tabular}
}

\afterpage{\null\newpage}

%--------------------------------------------------------------------------

%\paginadedicatoria
%\paginaagradecimento
%\paginaepigrafe
% \imprimirpaginaresumo
\par Multiprocessadores em chip demandam o uso de protocolos para assegurar a consistência de
memória compartilhada e a coerência de cache, os quais são implementados via hardware. A
crescente complexidade dessas implementações torna o projeto de sistemas de memória suscetível
a erros. Para endereçar esse problema, técnicas de verificação pré-silício têm sido estudadas
no meio acadêmico. Por serem executadas em simuladores de uma plataforma real, essas técnicas
não viabilizam a execução de testes demasiado extensos. Pensando nisso, algumas abordagens de
geração adaptativa foram propostas, procurando maximizar a cobertura funcional dos testes sem
aumentar a extensão dos mesmos. Neste trabalho é proposta a elaboração de uma nova técnica de
geração automática de testes adaptativos. A técnica será concebida a fim de aumentar a eficácia
em se expor erros de projeto no subsistema de memória compartilhada, em especial erros de
coerência, e manter-se independente da organização da memória, permitindo o reuso em diversos
sistemas.
%\paginaabstract
%\pretextuais % Substitui todos os elementos pre-textuais acima
%\listadefiguras % as listas dependem da necessidade do usuário
%\listadetabelas
%\listadeabreviaturas

%\begin{singlespacing}
%\input{init/acronym}
%\end{singlespacing}

%\acrodefplural{pmd}[PMDs]{Dispositivos Móveis Portáteis - \textit{Portable Mobile Devices}}%\acp{mv}\acp{pmd}

%\listadesimbolos
%\critico{mudar a formatação da lista de símbolos}
\newpage
\tableofcontents

%--------------------------------------------------------------------------
\chapter{Introdução}

No contexto de um multiprocessador em chip com memória compartilhada, o uso de caches privadas
torna necessário implementar um protocolo de coerência para garantir que cópias antigas de
blocos de cache sejam invalidadas quando um núcleo de processamento atualiza sua própria cópia.
A programação paralela de propósito geral conta com o gerenciamento implícito da coerência de
memória (via hardware), justificando a necessidade de uma abstração de memória compartilhada
coerente para multiprocessadores em chip, mesmo em face a um grande número de núcleos \cite{Devadas:2013}.

Além disso, operações de memória podem executar fora da ordem de programa. Tal comportamento é
definido no que se chama \textit{modelo de memória}, o qual especifica regras de consistência
que definem tanto o grau de relaxação da ordem de programa quanto a extensão da atomicidade de
escrita \cite{Adve:1996}. Por grande parte dos programas serem sincronizados, um programador
comum acaba não tomando consciência das regras de consistência \cite{Hennessy:2011}, o que
permite o uso de modelos de memória mais relaxados, alcançando maior performance sem prejudicar
o programador.

Essas características aumentam a complexidade do hardware, demandando a aplicação de técnicas
de verificação para detectar possíveis erros. A verificação pré-silício se baseia na execução
de testes sobre uma simulação do sistema de memória com base na representação do
\textit{design} de sua implementação. Isso permite validar um projeto sem a necessidade de
sintetizar protótipos para a execução dos testes, como é o caso da verificação pós-silício.

Dentre os verificadores pré-silício, existem dois tipos: os \textit{post-mortem checkers} e os
\textit{runtime checkers}. O primeiro trata da detecção de erros após o programa de teste
finalizar sua execução, como é o caso do \textit{checker} relatado em \citeonline{Rambo:2012}).
O último, por outro lado, realiza a análise do programa durante sua execução, agilizando o
processo de detecção. Os \textit{checkers} relatados em \citeonline{Shacham:2008} e
\citeonline{Freitas:2013}) são exemplos de \textit{runtime checkers}, com destaque ao último,
pois oferece garantias comprovadas, não acusando qualquer falso-positivo ou falso-negativo.

A qualidade do verificador não é o único fator que influencia na verificação. Esses
verificadores dependem de um mecanismo de geração de programas de teste, os quais devem
estimular situações de concorrência por meio de operações de leitura e escrita em endereços
compartilhados por múltiplos núcleos. Nesse sentido, a geração pseudoaleatória de testes para
verificação de modelos de memória tem sido usada tanto para verificadores pré-silício
\cite{Shacham:2008, Freitas:2013} quanto pós-silício \cite{Hangal:2004, Manovit:2006:c,
Roy:2006,Hu:2012}.

Como a verificação pré-silício baseia-se na simulação do projeto real, a taxa de execução de
operações de memória é ordens de magnitude inferior ao protótipo em hardware, limitando a
viabilidade da execução de testes muito grandes. Para diminuir o tamanho dos testes sem afetar
a cobertura dos mesmos, foram propostas técnicas de geração de testes adaptativa
\cite{Wagner:2008m, Elver:2016}, de forma a guiar a criação de um novo teste com base nas
informações dos testes anteriores. Neste trabalho, propõe-se a criação de uma nova técnica de
geração de testes adaptativa.

%a fim de aumentar a eficácia em se expor erros de projeto no
%subsistema de memória compartilhada distribuída, em especial erros de coerência, mantendo-se
%independente da organização da memória, para poder ser reusado em diferentes sistemas.

%--------------------------------------------------------------------------
\chapter{Objetivos}

\section{Objetivo Geral}
O escopo deste trabalho é a verificação pré-silício de consistência de memória compartilhada e
coerência de cache em subsistemas de memória associados a um multiprocessador em chip. O
objetivo geral é desenvolver uma técnica adaptativa original para geração automática de testes, os quais serão executados em uma dada representação de multiprocessador.

\section{Objetivos Específicos}
\begin{itemize}
	\item \textbf{Aumentar a cobertura:} a geração adaptativa dos testes deve dirigir os testes de forma a maximizar a cobertura do sistema.
	\item \textbf{Manter a reusabilidade:} permitir que a técnica possa ser aplicada em diferentes sistemas de memória.
	\item \textbf{Comparar com outras técnicas:} a técnica proposta será comparada com outras técnicas já implementadas e que se encontram em domínio público ou na infraestrutura disponibilizada pelo laboratório.
\end{itemize}

%--------------------------------------------------------------------------
\section{Método de Pesquisa}

A técnica será desenvolvida em uma linguagem de programação de propósito geral, definida
através de uma meta-heurística com função de custo associada. Inicialmente, será construída uma
prova de conceito, produzindo um protótipo com função custo e meta-heurística simples (e.g.
operações conflitantes e melhoria iterativa). Essa técnica terá sua eficácia e cobertura
comparadas com a técnica de geração pseudoaleatória pura. Com base nesses dados preliminares,
será realizado um refinamento na técnica, melhorando sua função de custo e meta-heurística.

Sobre a técnica refinada, serão realizados experimentos extensivos, a fim de comparar com
outras técnicas. Os experimentos serão realizados por meio do simulador \textit{gem5},
utilizando como modelo de memória o \textit{Alpha}, três níveis de cache e coerência baseada em
diretório. Os erros de projeto artificiais e \textit{checkers} utilizados serão os disponíveis
no acervo do laboratório.

%--------------------------------------------------------------------------
\chapter{Cronograma}
% \noindent{
% 	\begin{minipage*}{1\textwidth}
% 	\centering
\noindent\resizebox{\textwidth}{!}{

	\rowcolors{0}{shadecolor}{shadecolor}
	\begin{tabular}{|X p{3cm}|c|c|c|c|c|c|c|c|c|c|c|c|}
		%\begin{small}
		\hline
		%{\rowcolor{shadecolor}}
		%\multirow{2}{*}{\cellcolor{shadecolor}} \\
		\multirow{-1}{*}{Etapas} &
		%\cline{2-12}
		\multicolumn{12}{|c|}{Meses} \\ \hhline{|~|*{12}{-|}}
		%{\rowcolor{shadecolor}}
		& jan. & fev. & mar. & abr. & mai. & jun. & jul. & ago. & set. & out. & nov. & dez. \\ \hline
		\hiderowcolors Revisão do estado da arte 												& x    & x	  &      &      &      &      &      &      &      &      &      &		\\ \hline
		\hiderowcolors Desenvolvimento do protótipo
		&      & x    & x    &      &      &      &      &      &      &      &      &      \\
		\hline
		\hiderowcolors Testes e comparação preliminares
		&  	   &      & x    & x    &      &      &      &      &      &      &      &		\\
		\hline
		\hiderowcolors Refinamento da técnica
		&      &      &      & x    & x    &      &      &      &      &      &      &		\\
		\hline
		\hiderowcolors Experimentação extensiva com a nova técnica
		&  	   &  	  &      &      & x    & x    & x    &      &      &      &      &		\\
		\hline
		\hiderowcolors Comparação com outras técnicas*
		&  	   &  	  &      &      &      &      & x    & x    &      &      &      &		\\
		\hline
		\hiderowcolors Redação do rascunho da monografia
		&  	   &  	  &      &      &      &      &      &      & x    & x    &      &		\\
		\hline
		\hiderowcolors Preparação da defesa
		&  	   &   	  &      &      &      &      &      &      &      & x    & x    &		\\
		\hline
		\hiderowcolors Revisão final da monografia
		&  	   &  	  &      &      &      &      &      &      &      &      & x    & x
		\\ \hline
		\hiderowcolors Entrega do relatório de TCC 1
		&  	   &  	  &      &      & o    &      &      &      &      &      &      &		\\
		\hline
		\hiderowcolors Entrega do rascunho da monografia
		&  	   &  	  &      &      &      &      &      &      &      & o    &      &		\\
		\hline
		\hiderowcolors Defesa do TCC
		&  	   &  	  &      &      &      &      &      &      &      &      & o    &		\\
		\hline
	\end{tabular}
	\footnotetext{Por técnicas entende-se aquelas já disponíveis na infraestrutura do laboratório e outras que se apresentem implementação em domínio público}
}
\textit{* Por técnicas entende-se aquelas já disponíveis na infraestrutura do laboratório e outras que apresentem implementação em domínio público}
% \end{minipage*}
% }\vspace{-.1cm}

\chapter{Custos}
\vspace{-.1cm}
% \noindent\resizebox{\textwidth}{!}{
\begin{tabular}{|X p{3cm}|c|c|c|}
	%\begin{small}
	\hline
	%\rowcolor{shadecolor}
	{\cellcolor{shadecolor}} Item & {\cellcolor{shadecolor}} Quant. & {\cellcolor{shadecolor}} Valor Un. (R\$) & {\cellcolor{shadecolor}} Total (R\$) \\ \hline
	\hline
	\multicolumn{4}{|c|}{Material de Consumo} \\ \hline
	CD & 6 & $1,50$ & $9,00$ \\ \hline
	Folhas Impressas & 500 & $0,20$ & $100,00$ \\ \hline
	%		    \hline
	%		    \multicolumn{4}{|c|}{Reserva de Contingência} \\ \hline
	%		     &  &  & $1500,00$ \\ \hline
	%		    \hline
	%		    \multicolumn{4}{|c|}{Total} \\ \hline
	%		     &  &  & $6586,00$ \\ \hline

	%\end{small}
\end{tabular}
% }

\chapter{Recursos Humanos}

% \noindent\resizebox{\textwidth}{!}{
\begin{tabular}{|c|c|}
	%\begin{small}
	\arrayrulecolor{white}
	\hline
	\arrayrulecolor{black}
	\hline
	\rowcolor{shadecolor}
	Nome & Função \\ \hline
	Marleson Graf & Autor \\ \hline
	Luiz Claudio Villar dos santos & Orientador/Professor Responsável \\ \hline
	Djones Vinicius Lettnin & Membro da Banca \\ \hline
	Laércio Lima Pilla & Membro da Banca \\ \hline
	Renato Cislaghi & Coordenador de Projetos \\ \hline

	%\end{small}
\end{tabular}
% }


\chapter{Comunicação}

\noindent\resizebox{\textwidth}{!}{
	\begin{tabular}{|X p{3cm}|X p{3cm}|X p{3cm}|X p{3cm}|X p{3cm}|}
		%\begin{small}
		%\arrayrulecolor{white}
		%\hline
		%\arrayrulecolor{black}
		\hhline{|*{5}{-|}}
		%\rowcolor{shadecolor}
		%\rowcolor{white}
		{\cellcolor{shadecolor}} O que precisa ser comunicado & {\cellcolor{shadecolor}} Por quem & {\cellcolor{shadecolor}} Para quem & {\cellcolor{shadecolor}} Melhor forma de Comunicação & {\cellcolor{shadecolor}} Quando e com que frequência \\ \hhline{|*{5}{-|}}
		%\rowcolor{white}
		Entregas & Autor & Coordenador de Projetos & site do TCC & Nos dias estipulados pelas disciplinas\\ \hhline{|*{5}{-|}}
		%\rowcolor{white}
		Reuniões com o orientador & Autor & Orientador & presencial & Quinzenalmente \\ \hhline{|*{5}{-|}}
		%\rowcolor{white}
		Revisões da monografia & Autor & Orientador, membros da banca & papel impresso ou pdf & período de elaboração da monografia\\ \hhline{|*{5}{-|}}
		%\rowcolor{white}
		Dúvidas & Autor & Orientador, Membros da Banca ou Coordenador de Projetos & e-mail e/ou presencial & Quando necessário\\ \hhline{|*{5}{-|}}

		%\end{small}
	\end{tabular}
}

\chapter{Riscos}

\noindent\resizebox{\textwidth}{!}{
	\begin{tabular}{|X p{2cm}|c|c|c|X p{3cm}|X p{3cm}|}
		%\begin{small}
		%\hline
		\hhline{|*{6}{-|}}
		%\rowcolor{shadecolor}
		{\cellcolor{shadecolor}} Risco & {\cellcolor{shadecolor}}Probabilidade & {\cellcolor{shadecolor}}Impacto & {\cellcolor{shadecolor}}Prioridade & {\cellcolor{shadecolor}}Estratégia de Resposta & {\cellcolor{shadecolor}}Ações Preventivas \\ \hhline{|*{6}{-|}}
		Perda de dados (HD) & média & alto & alta & Recuperação dos dados e aquisição de novo HD & Gerar backups periodicamente\\ \hhline{|*{6}{-|}}
		Alteração no tema & baixa & alto & alta & Modificar o escopo do tema ou adotar um novo tema & Manter interação constante com orientador\\ \hhline{|*{6}{-|}}
		Alteração no cronograma & baixa & alto & média & Diminuir o escopo do trabalho & Monitorar continuamente as informações obtidas dos superiores imediatos \\ \hhline{|*{6}{-|}}
		Problemas de saúde & baixa & média & média & Realizar tratamento e retomar as atividades o quanto antes & Ter hábitos saudáveis e ser precavido\\ \hhline{|*{6}{-|}}

		%\end{small}
	\end{tabular}
}

\bibliographystyle{ufscThesis/ufsc-alf}
\bibliography{research}

%--------------------------------------------------------
% Elementos pós-textuais
%\apendice
%\chapter{Exemplificando um Apêndice}
%Texto do Apêndice aqui.
\begin{anexosenv}
\anexos
\chapter{DECLARAÇÃO PADRÃO PARA EMPRESA OU LABORATÓRIO}

\begin{snugshade}

\begin{center}
{\textbf{DECLARAÇÃO DE CONCORDÂNCIA COM AS CONDIÇÕES PARA O DESENVOLVIMENTO DO TCC NA INSTITUIÇÃO}}
\end{center}

\end{snugshade}

\vspace{10pt}

Declaro estar ciente das premissas para a realização de Trabalhos de Conclusão de Curso (TCC) de Ciência da Computação e Sistema de In\-for\-ma\-ções da UFSC, particularmente da necessidade de que se o TCC envolver o desenvolvimento de um software ou produto específico (ex: um protocolo, um método computacional, etc.) o código fonte e/ou documentação completa correspondente deverá ser entregue integralmente, como parte integrante do relatório final do TCC.

Ciente dessa condição básica, declaro estar de acordo com a realização do TCC identificado pelos dados apresentados a seguir.

\vspace{20pt}


\noindent\resizebox{\textwidth}{!}{
	\begin{tabular}{|l|X p{8cm}|}
		%\begin{small}
			\hline
   \textbf{Instituição} &  ECL/INE/CTC \\ \hline
   \textbf{Nome do Responsável} &  Luiz Claudio Villar dos Santos \\ \hline
   \textbf{Cargo/Função} &  Prof. INE/CTC \\ \hline
   \textbf{Fone de Contato} & (48) 3721 7549\\ \hline
   \textbf{Acadêmico} & Marleson Graf \\ \hline
   \textbf{Título do trabalho} & Verificação pré-silício de memória baseada em testes dirigidos adaptativos\\ \hline
   \textbf{Curso} & Ciências da Computação/INE/UFSC \\ \hline

		%\end{small}
	\end{tabular}
}

\vspace{40pt}

\begin{flushright}

	Florianópolis, 14 de Dezembro de 2015.

\end{flushright}

\vspace{20pt}


\begin{center}
 \parbox{7cm}{%
 \centering
   \rule{6cm}{1pt}\\
    \small \textbf{Professor Responsável}\\
    Prof. Dr. Luiz Claudio Villar dos Santos
 }
	\hfill
\end{center}

\end{anexosenv}

\end{document}

