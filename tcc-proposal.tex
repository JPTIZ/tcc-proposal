\documentclass{ufsc-thesis}
\usepackage[alf]{abntex2cite}

%----------------------------------------------------------------------
% Pacotes usados especificamente neste documento
\usepackage{graphicx}
\usepackage{color}
\usepackage{listings}
\usepackage{multirow}
\usepackage{tabularx}
\usepackage[table]{xcolor}
\usepackage{colortbl}
\usepackage{framed}
\usepackage{footnote}
\makesavenoteenv{tabular}
\makesavenoteenv{table}
\usepackage[printonlyused, withpage]{acronym}
\usepackage{pdfpages}
\usepackage{afterpage}
\usepackage{hhline}
\usepackage{enumitem}
\usepackage{latexsym}
\usepackage{bm}
\usepackage{amssymb}
\usepackage[normalem]{ulem}
\usepackage[lined,boxed,ruled,commentsnumbered,portuguese]{algorithm2e}

\usepackage{fontspec}
\setmainfont{times.ttf}[
    BoldFont = timesbd.ttf,
    ItalicFont = timesi.ttf,
    BoldItalicFont = timesbi.ttf
]

\usepackage{tikz}
\usetikzlibrary{decorations.pathreplacing}
\usepackage{standalone}

%----------------------------------------------------------------------
% Comandos criados pelo usuário
%----------------------------------------------------------------------
\newcommand{\todo}[1]{{\color{red}{#1}}}
\newcommand{\critical}[1]{{\color{red}\textbf{{#1}}}}
\newcommand{\verycritical}[1]{{\color{red}\textbf{\uppercase{{#1}}}}}

\definecolor{shadecolor}{rgb}{0.8,0.8,0.8}
\newcommand\VRule[1][\arrayrulewidth]{\vrule width #1}

\newcommand{\shadecell}{{\cellcolor{shadecolor}}}

\usepackage[a5paper,inner=2.5cm,outer=1.5cm,top=2.0cm,bottom=1.5cm,head=0.7cm,foot=0.7cm]{geometry}
\renewcommand{\normalsize}{\small}

%----------------------------------------------------------------------
% Identificadores do trabalho
% Usados para preencher os elementos pré-textuais
%----------------------------------------------------------------------
\titulo{Limitações das Técnicas de Geração Automática de Programas de Teste para a
Verificação de Memória Compartilhada em Multi-cores}
\autor{João Paulo Taylor Ienczak Zanette}
\data{\today}
\instituicao{Universidade Federal de Santa Catarina}
\local{Florianópolis}
\tipotrabalho{Trabalho de Conclus\~{a}o de Curso}
\orientador{Prof. Dr. Luiz Claudio Villar dos Santos}
\programa{Curso de Bacharelado em Ciências da Com\-pu\-ta\-ção}
\centro{Departamento de Informática e Estatística}

\def\bancaMembroA{\todo{A definir}}
\def\bancaMembroB{\todo{A definir}}

%----------------------------------------------------------------------
% Preâmbulo
%----------------------------------------------------------------------
\preambulo{Trabalho de Conclusão de Curso submetido ao Curso de Bacharelado em
           Ciências da Computação para a obtenção do Grau de Bacharel
           em Ciências da Computação.}
\assuntos{Ciências da Computação,Modelos,Teses,OpenSource,LaTeX}

\assuntos{EDA, verificação, pré-silício, memória compartilhada, multicore, coerência de cache}

\renewcommand{\imprimircapa}{%
    \begin{capa}%
        \center
        {\ABNTEXchapterfont\large\MakeUppercase{\imprimirinstituicao}\\
        \ABNTEXchapterfont\large\MakeUppercase{\imprimircentro}}

        \vspace*{1cm}

        {{\normalfont\large\imprimirautor}}

        \vspace*{4cm}
        \begin{center}
            \ABNTEXchapterfont\bfseries\large\MakeUppercase{\imprimirtitulo}
        \end{center}
        \vfill

        \large\imprimirlocal\\
        \large\the\year
        \vspace*{1cm}
    \end{capa}
}

%--------------------------------------------------------------------------
% Início do documento

\begin{document}
%--------------------------------------------------------------------------
% Elementos pré-textuais
\pretextual
\imprimircapa

\imprimirfolhaderosto

\afterpage{\null\newpage}

%--------------------------------------------------------------------------
% folha de aprovação de proposta de TCC
\begin{snugshade}
    \begin{center}
        {\textbf{\small{FOLHA DE APROVAÇÃO DE PROPOSTA DE TCC}}}
    \end{center}
\end{snugshade}

\vspace{-8pt}
\noindent\resizebox{\textwidth}{!}{
    \footnotesize
    \begin{tabular}{|l|X p{8cm}|}
        \hline
        \textbf{Acadêmico} &  \imprimirautor \\ \hline
        \textbf{Título do trabalho} & \imprimirtitulo \\ \hline
        \textbf{Curso} & Ciências da Computação/INE/UFSC \\ \hline
        \textbf{Área de Concentração} &  \textit{Hardware} \\ \hline
    \end{tabular}
}

\vspace{8pt}

{%
    \small
    \noindent\textbf{Instruções para preenchimento pelo \uline{ORIENTADOR DO TRABALHO}:}
    \begin{itemize}[leftmargin=*,noitemsep,topsep=0pt]
        \item[-] Para cada critério avaliado, assinale um X na coluna SIM
            apenas se considerado aprovado. Caso contrário, indique as
            alterações necessárias na coluna Observação.
    \end{itemize}
}

\vspace{8pt}

\noindent\resizebox{\textwidth}{!}{%
    \footnotesize
    \begin{tabular}{|X p{6cm}|X p{0.5cm}|X p{0.8cm}|X p{0.5cm}|X p{1cm}|X p{3.2cm}|}
        \hline
        \shadecell & \multicolumn{4}{c|}{\shadecell \textbf{Aprovado}} & \shadecell \\ \hhline{*{1}{>{\arrayrulecolor{shadecolor}}-}*{4}{>{\arrayrulecolor{black}}|-}>{\arrayrulecolor{shadecolor}}|->{\arrayrulecolor{black}}}
        \multirow{-1}{*}{\shadecell \textbf{Critérios}} & \shadecell\textbf{Sim} &  \shadecell\textbf{Parcial} & \shadecell\textbf{Não} & \shadecell\textbf{Não se aplica} & \multirow{-1}{*}{\shadecell \textbf{Observação}} \\ \hline
        1.O trabalho é adequado para um TCC no CCO/SIN (relevância/abrangência)? & \shadecell  & \shadecell & \shadecell  & \shadecell  & \\ \hline
        2.O título do trabalho é adequado? & \shadecell & \shadecell & \shadecell & \shadecell & \\ \hline
        3.O tema de pesquisa está claramente descrito? & \shadecell & \shadecell & \shadecell  & \shadecell & \\ \hline
        4.O problema/hipóteses de pesquisa do trabalho está claramente identificado? & \shadecell  & \shadecell  & \shadecell & \shadecell & \\ \hline
        5.A relevância da pesquisa é justificada? & \shadecell & \shadecell & \shadecell  & \shadecell & \\ \hline
        6.Os objetivos descrevem completa e claramente o que se pretende alcançar neste trabalho? & \shadecell & \shadecell & \shadecell & \shadecell & \\ \hline
        7.É definido o método a ser adotado no trabalho? O método condiz com os objetivos e é adequado para um TCC? & \shadecell & \shadecell & \shadecell & \shadecell & \\ \hline
        8.Foi definido um cronograma coerente com o método definido (indicando todas as atividades) e com as datas das entregas (p.ex.Projeto I, II, Defesa)? & \shadecell & \shadecell & \shadecell & \shadecell & \\ \hline
        9.Foram identificados custos relativos à execução deste trabalho (se houver)? Haverá financiamento para estes custos? & \shadecell & \shadecell & \shadecell & \shadecell & \\ \hline
        10.Foram identificados todos os envolvidos neste trabalho? & \shadecell & \shadecell & \shadecell & \shadecell & \\ \hline
        11.As formas de comunicação foram definidas (ex: horários para orientação)? & \shadecell  & \shadecell & \shadecell & \shadecell & \\ \hline
        12.Riscos potenciais que podem causar desvios do plano foram identificados? & \shadecell  & \shadecell & \shadecell & \shadecell & \\ \hline
        13.Caso o TCC envolva a produção de um software ou outro tipo de produto e seja desenvolvido também como uma atividade  realizada numa empresa ou laboratório, consta da proposta uma declaração (\anexoname\ \ref{decl-concor}) de ciência e concordância com a entrega do código fonte e/ou documentação produzidos? & \shadecell & \shadecell & \shadecell & \shadecell & \\ \hline
    \end{tabular}
}

\vspace{12pt}


\noindent\resizebox{\textwidth}{!}{
    \scriptsize
    \begin{tabular}{|X p{3cm}|X p{2.35cm}|X p{1.6cm}|X p{3.4cm}|}
        \hline
        \textbf{Avaliação} & \multicolumn{1}{l}{\textbf{$\Box$ \footnotesize Aprovado}} & \multicolumn{2}{c|}{\textbf{$\Box$ \footnotesize Não Aprovado}} \\ \hline \hline
        \textbf{Professor Responsável} &  {\footnotesize \imprimirorientador} & & \\ \hline
        \textbf{Orientador} & {\footnotesize \imprimirorientador} & & \\ \hline
    \end{tabular}
}

\clearpage
\afterpage{\null\newpage}

%--------------------------------------------------------------------------
\begin{resumo}
    \small
    Multiprocessadores em chip demandam o uso de protocolos para assegurar a
    consistência de memória compartilhada e a coerência de cache, os quais são
    implementados via hardware. A crescente complexidade dessas implementações
    torna o projeto de sistemas de memória suscetível a erros. Para endereçar
    esse problema, técnicas de verificação pré-silício têm sido estudadas no
    meio acadêmico. Por serem executadas em simuladores de uma plataforma real,
    essas técnicas não viabilizam a execução de testes demasiado extensos.
    Pensando nisso, algumas abordagens de geração adaptativa foram propostas,
    procurando maximizar a cobertura funcional dos testes sem aumentar a
    extensão dos mesmos. Neste trabalho é proposta a elaboração de uma nova
    técnica de geração automática de testes adaptativos. A técnica será
    concebida a fim de aumentar a eficácia em se expor erros de projeto no
    subsistema de memória compartilhada, em especial erros de coerência, e
    manter-se independente da organização da memória, permitindo o reuso em
    diversos sistemas.

    \vspace{\onelineskip}
    \noindent
    \textbf{Palavras-chave}: \listaassuntos
\end{resumo}

\afterpage{\null\newpage}

\begin{KeepFromToc}
    \tableofcontents
\end{KeepFromToc}

%--------------------------------------------------------------------------
\chapter{Introdução}

Depois de ter provocado uma mudança radical na forma de se projetar
microprocessadores para PCs e servidores, o multi-processamento em
\textit{chip} ou \textit{Chip Multi-Processing (CMP)} levou ao atual quadro
onde \textit{chips} com 8 ou mais \textit{cores} são bastante comuns. Prevê-se
o uso de \textit{chips} com dezenas a centenas cores e a sobrevida da abstração
de memória compartilhada como condição para viabilizar a programação de
propósitos gerais \cite{Devadas:2013}. Isso resulta no desafio de como manter
coerência de memória compartilhada \textbf{quando se aumenta a escala do número
de cores em um único \textit{chip}}. Para suprir a abstração de memória
compartilhada coerente, o subsistema de memória torna-se mais sofisticado e,
portanto, mais suscetível a erros de projeto com o aumento massivo do número de
\textit{cores}. Além disso, como se espera que grande parte dos programas
paralelos utilize bibliotecas para sincronização, a maioria dos programadores
não precisa se preocupar com as regras de consistência \cite{Adve:1996} do
modelo de memória \cite{Hennessy:2011}, o que tende a popularizar o uso de
\textbf{modelos com máxima relaxação da ordem de programa} para aumentar o
desempenho sem comprometer a programabilidade.  Isso também contribui para
aumentar a complexidade do hardware.

Portanto, a dificuldade de se validar sistemas computacionais baseados em CMP
tende a aumentar dramaticamente a cada nova geração de produtos a serem
lançados. Ora, \textbf{a validação da coerência e da consistência do subsistema
de memória compartilhada}, o qual inclui múltiplos níveis de cache e protocolos
complexos, constitui grande parte do esforço de se validar um sistema
computacional baseado em CMP \cite{Wagner:2008}. Como o número de estados
induzidos por um protocolo de coerência cresce exponencialmente com o aumento
do número de cores \cite{Shim:2013}, torna-se bastante desafiador o problema de
verificar se o projeto de um chip baseado em múltiplos \textit{cores} em larga
escala implementa corretamente o comportamento esperado para o subsistema de
memória compartilhada.

As primeiras técnicas de validação do sistema de memória foram propostas para
serem aplicadas após a fabricação (\textbf{teste} pós-silício), através da
execução de longos programas de testes aleatórios no próprio hardware do
protótipo.  Infelizmente, o simples reuso dessas técnicas é inadequado em tempo
de projeto (para \textbf{verificação} pré-silício), porque seriam
demasiadamente lentas quando os programas de teste são executados em
simuladores, o que requer a limitação do tamanho do teste, o que restringe a
qualidade da verificação. Por isso, as técnicas de verificação tem se mostrado
frequentemente incapazes de encontrar erros sutis de projeto, que acabam
passando para o hardware.

%--------------------------------------------------------------------------
\chapter{Objetivos}

\section{Objetivo Geral}
O objetivo geral desta proposta é o estudo de técnicas de geração
de testes mais eficientes e eficazes, a serem aplicadas em tempo de projeto,
para solucionar o problema de verificação de consistência e coerência de
memória compartilhada em sistemas computacionais baseados em CMP.

% O primeiro objetivo específico desta proposta é o de investigar as principais
% limitações das técnicas reportadas na literatura, a fim de identificar
% oportunidades de contribuição científica.

% O segundo objetivo específico é a implementação de um protótipo de gerador de
% testes dirigidos reportado na literatura (e.g. [WAG 08] [ELV 16].) e sua
% comparação com um gerador convencional de testes aleatórios e com um gerador de
% testes aleatórios [RAM 11] sob fortes restrições [AND 16a].

\section{Objetivos Específicos}
\begin{itemize}
    \item \textbf{Analisar técnicas existentes:} investigar as principais
        limitações das técnicas reportadas na literatura, a fim de identificar
        oportunidades de contribuição científica.
    \item \textbf{Implementar um gerador de testes dirigidos:} tomar como base um
        gerador reportado na literatura (e.g. \cite{Wagner:2008},
        \cite{Elver:2016}).
    \item \textbf{Comparar com outros geradores:} comparar o gerador implementado
        com um convencional de testes aleatórios \cite{Rambo:2011} e com um de
        testes aleatórios sob fortes restrições \cite{Andrade:2016a}.
\end{itemize}

%--------------------------------------------------------------------------
\section{Método de Pesquisa}

O método a ser adotado consiste na realização de experimentos sobre uma
\textbf{representação de projeto do sistema sob verificação}, ou seja, na
execução de testes através da simulação daquela representação, utilizando como
infraestrutura um \textbf{simulador} de domínio público denominado gem5
\cite{Gem5:2012}. A representação de projeto utilizada para o subsistema de
memória corresponde ao módulo \textit{Ruby} daquele simulador, a qual permite a
descrição das máquinas de estado dos protocolos de coerência.

Para desempenhar o papel de \textbf{verificador} da correção funcional do
subsistema de memória representado, será utilizada uma técnica de análise
automática dos eventos registrados durante a simulação, a qual foi desenvolvida
localmente \cite{Freitas:2013} e cujo código está portanto disponível para uso
no laboratório hospedeiro.

O protótipo de gerador de testes dirigidos será implementado com base na
descrição de seu algoritmo conforme descrito na literatura. Será escolhido o
gerador cuja descrição for mais completa e leve à melhor reprodutibilidade.

Como representante das técnicas de geração convencional de testes aleatórios
será utilizado o gerador denominado PLAIN \cite{Rambo:2011}. Como representante
das técnicas de geração convencional de testes aleatórios sob fortes restrições
será utilizado o gerador denominado CHAIN \cite{Andrade:2016a}. Os códigos dos
dois geradores estão disponíveis para uso no laboratório hospedeiro.

Os três geradores serão comparados ao se usar a mesma representação de projeto,
o mesmo simulador e o mesmo verificador acima mencionados.

Para a simulação de erros de projeto, serão utilizados como base os cinco erros
artificiais descritos em \cite{Andrade:2016b}, mas novos erros artificiais
serão desenvolvidos como parte do trabalho proposto para aumentar a abrangência
da comparação.

Para a comparação serão adotadas as seguintes métricas: cobertura funcional,
esforço computacional e probabilidade de detecção de erros.

%--------------------------------------------------------------------------
\chapter{Cronograma}

\noindent\resizebox{\textwidth}{!}{

    \rowcolors{0}{shadecolor}{shadecolor}
    \begin{tabular}{|X p{3cm}|c|c|c|c|c|c|c|c|c|c|c|c|}
        \hline
            \multirow{-1}{*}{Etapas} &
    
            \multicolumn{12}{|c|}{Meses} \\ \hhline{|~|*{12}{-|}}
            
            % [construir um cronograma de dois semestres alocando um total de
            
            % - três mêses:
            %       atividades de 1 a 5 (superpostas);
            
            % - um mês:
            %       atividade 6 (sem sobreposição com nenhuma outra)
            
            % - 6 meses:
            %       atividade 7 (sem sobreposição com nenhuma outra)
            
            % - um mês:
            %       atividades de 8 a 11 (superpostas)]
    
            & jul. & ago. & set. & out. & nov. & dez. & jan. & fev. & mar. & abr. & mai. & jun. \\ \hline
            \hiderowcolors Estudo dos modelos de consistência de memória
            & x    &      &      &      &      &      &      &      &      &      &      &      \\ \hline
            \hiderowcolors Estudo de um protocolo de coerência (MESI)
            & x    & x    &      &      &      &      &      &      &      &      &      &      \\ \hline
            \hiderowcolors Estudo de geradores de testes aleatórios e testes dirigidos
            & x    & x    & x    &      &      &      &      &      &      &      &      &      \\ \hline
            \hiderowcolors Familiarização com o simulador (gem5)
            &      & x    & x    &      &      &      &      &      &      &      &      &      \\ \hline
            \hiderowcolors Familiarização com a representação de memória (Ruby)
            &      &      & x    &      &      &      &      &      &      &      &      &      \\ \hline
            \hiderowcolors Criação e caracterização de novos erros artificiais de projeto
            &      &      &      & x    &      &      &      &      &      &      &      &      \\ \hline
            \hiderowcolors Implementação do protótipo do gerador de testes dirigidos
            &      &      &      &      & x    & x    & x    & x    & x    & x    &      &      \\ \hline
            \hiderowcolors Execução de testes produzidos com o gerador implementado
            &      &      &      &      &      &      &      &      &      &      & x    &      \\ \hline
            \hiderowcolors Execução de testes produzidos com o gerador PLAIN
            &      &      &      &      &      &      &      &      &      &      & x    &      \\ \hline
            \hiderowcolors Execução de testes produzidos com o gerador CHAIN
            &      &      &      &      &      &      &      &      &      &      & x    &      \\ \hline
            \hiderowcolors Comparação dos geradores de acordo com as métricas
            &      &      &      &      &      &      &      &      &      &      & x    &      \\ \hline
            \hiderowcolors Entrega do relatório de TCC I
            &      &      &      &      & o    &      &      &      &      &      &      &      \\ \hline
            \hiderowcolors Entrega do rascunho da monografia
            &      &      &      &      &      &      &      &      &      &      & o    &      \\ \hline
            \hiderowcolors Defesa do TCC
            &      &      &      &      &      &      &      &      &      &      &      & o    \\ \hline
        \end{tabular}
        \footnotetext{Por técnicas entende-se aquelas já disponíveis na
        infraestrutura do laboratório e outras que se apresentem implementação
        em domínio público}
}

\textit{* Por técnicas entende-se aquelas já disponíveis na
infraestrutura do laboratório e outras que apresentem implementação em
domínio público}


\chapter{Custos}

\vspace{-.1cm}
\begin{tabular}{|X p{3cm}|c|c|c|}
    \hline
    {\shadecell} Item & {\shadecell} Quant. & {\shadecell} Valor Un. (R\$) & {\shadecell} Total (R\$) \\ \hline
    \hline
    \multicolumn{4}{|c|}{Material de Consumo} \\ \hline
    CD & 6 & $1,50$ & $9,00$ \\ \hline
    Folhas Impressas & 500 & $0,20$ & $100,00$ \\ \hline
\end{tabular}


\chapter{Recursos Humanos}

\begin{tabular}{|l|l|}
    \arrayrulecolor{white}
    \hline
    \arrayrulecolor{black}
    \hline
    \rowcolor{shadecolor}
    Nome & Função \\ \hline
    \imprimirautor & Autor \\ \hline
    \imprimirorientador & Orientador/Professor Responsável \\ \hline
    Gabriel Arthur Gerber Andrade & \todo{Ver função adequada} \\ \hline
    Marleson Graf & \todo{Ver função adequada} \\ \hline
    % \bancaMembroA{} & Membro da Banca \\ \hline
    % \bancaMembroB{} & Membro da Banca \\ \hline
    Renato Cislaghi & Coordenador de Projetos \\ \hline
\end{tabular}


\chapter{Comunicação}

\noindent\resizebox{\textwidth}{!}{
    \begin{tabular}{|X p{3cm}|X p{3cm}|X p{3cm}|X p{3cm}|X p{3cm}|}
        \hhline{|*{5}{-|}}
        {\shadecell} O que precisa ser comunicado & {\shadecell} Por quem & {\shadecell} Para quem & {\shadecell} Melhor forma de Comunicação & {\shadecell} Quando e com que frequência \\ \hhline{|*{5}{-|}}
        
        Entregas & Autor & Coordenador de Projetos & Site do TCC & Nos dias estipulados pelas disciplinas\\ \hhline{|*{5}{-|}}
        
        Reuniões com o orientador & Autor & Orientador & Presencial & Quinzenalmente \\ \hhline{|*{5}{-|}}
        
        Revisões da monografia & Autor & Orientador, membros da banca & Papel impresso ou pdf & Período de elaboração da monografia\\ \hhline{|*{5}{-|}}
        
        Dúvidas & Autor & Orientador, Membros da Banca ou Coordenador de Projetos & E-mail e/ou presencial & Quando necessário\\ \hhline{|*{5}{-|}}
    \end{tabular}
}

\chapter{Riscos}

\noindent\resizebox{\textwidth}{!}{
    \begin{tabular}{|X p{2cm}|c|c|c|X p{3cm}|X p{3cm}|}
        \hhline{|*{6}{-|}}
        {\shadecell} Risco & {\shadecell}Probabilidade & {\shadecell}Impacto & {\shadecell}Prioridade & {\shadecell}Estratégia de Resposta & {\shadecell}Ações Preventivas \\ \hhline{|*{6}{-|}}
        Perda de dados (HD) & Média & Alto & Alta & Recuperação dos dados e aquisição de novo HD. & Versionar desenvolvimento remotamente. Gerar backups periodicamente. \\ \hhline{|*{6}{-|}}
        Alteração no tema & Baixa & Alto & Alta & Modificar o escopo do tema ou adotar um novo tema. & Manter interação constante com orientador. \\ \hhline{|*{6}{-|}}
        Alteração no cronograma & Baixa & Alto & Média & Diminuir o escopo do trabalho. & Monitorar continuamente as informações obtidas dos superiores imediatos. \\ \hhline{|*{6}{-|}}
        Problemas de saúde & Baixa & Média & Média & Realizar tratamento e retomar as atividades o quanto antes. & Ter hábitos saudáveis e ser precavido. \\ \hhline{|*{6}{-|}}
    \end{tabular}
}

\bibliography{references}

%--------------------------------------------------------
% Elementos pós-textuais

\begin{anexosenv}
    \partanexos
    
    \chapter{DECLARAÇÃO PADRÃO PARA EMPRESA OU LABORATÓRIO}
    \label{decl-concor}

    \begin{snugshade}
        \begin{center}
            \textbf{DECLARAÇÃO DE CONCORDÂNCIA COM AS CONDIÇÕES PARA O DESENVOLVIMENTO DO TCC NA INSTITUIÇÃO}
        \end{center}
    \end{snugshade}

    \vspace{10pt}

    Declaro estar ciente das premissas para a realização de Trabalhos de Conclusão de Curso (TCC) de Ciência da Computação e Sistema de In\-for\-ma\-ções da UFSC, particularmente da necessidade de que se o TCC envolver o desenvolvimento de um software ou produto específico (ex: um protocolo, um método computacional, etc.) o código fonte e/ou documentação completa correspondente deverá ser entregue integralmente, como parte integrante do relatório final do TCC.

    Ciente dessa condição básica, declaro estar de acordo com a realização do TCC identificado pelos dados apresentados a seguir.

    \vspace{20pt}


    \noindent\resizebox{\textwidth}{!}{
        \begin{tabular}{|l|X p{8cm}|}
            \hline
            \textbf{Instituição} & ECL/INE/CTC \\ \hline
            \textbf{Nome do Responsável} & \imprimirorientador \\ \hline
            \textbf{Cargo/Função} & Prof. INE/CTC \\ \hline
            \textbf{Fone de Contato} & (48) 3721 7549\\ \hline
            \textbf{Acadêmico} & \imprimirautor \\ \hline
            \textbf{Título do trabalho} & Verificação pré-silício de memória baseada em testes dirigidos adaptativos\\ \hline
            \textbf{Curso} & Ciências da Computação/INE/UFSC \\ \hline
        \end{tabular}
    }

    \vspace{40pt}

    \begin{flushright}

        Florianópolis, 14 de Dezembro de 2015.

    \end{flushright}

    \vspace{20pt}


    \begin{center}
        \small
        \parbox{7cm}{%
            \centering
            \rule{6cm}{1pt}\\
            \textbf{Professor Responsável}\\
            \imprimirorientador
        }
        \hfill
    \end{center}

\end{anexosenv}

\end{document}

